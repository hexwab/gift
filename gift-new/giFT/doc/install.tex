% $Id: install.tex,v 1.22 2002/07/11 10:23:31 eelcol Exp $
% Copyright (c) 2001-2002, The giFT Project
% written by Eelco Lempsink (eelco@wideview.33lc0.net)

\documentclass[10pt]{article}

\usepackage{url}
\usepackage{hyperref}

\setlength{\parindent}{0pt}

\begin{document}

\begin{center}
\textsf{\textbf{\Huge{giFT}\\ \huge{Installation Guide}\\
\normalsize{Last update: \today}}}
\end{center}

\tableofcontents

\setlength{\parskip}{1.4ex}

\section{Introduction}
A small notice to begin with. Lots (and I mean lots) of people are
coming to our channel and ask about things covered in this
document. Please note that every line, every word and every letter
in this document actually matters. If you haven't read these
instructions completely, and start asking dumb things, you will be
shot.

\section{Getting it}

Because giFT/OpenFT has not yet been released, you must retrieve
the sources via CVS to use the program. If you don't have cvs
installed, do it now.

Go to the directory you want to download the sources to and type
(the \$ is used to indicate this is a shell command):

\begin{verbatim}
  $ cvs -d:pserver:anonymous@cvs.gift.sf.net:/cvsroot/gift login
\end{verbatim}

Just press enter when prompted for a password. You'll have to do
this only once, after that cvs will read your password information
from \~{}/.cvspass.  Now type:

\begin{verbatim}
  $ cvs -z3 -d:pserver:anonymous@cvs.gift.sf.net:/cvsroot/gift co giFT
\end{verbatim}

This will download everything you need...

\subsection{Updating}
To update giFT to include the latest updates from CVS, type:

\begin{verbatim}
  $ cd giFT
  $ cvs update -Pd
\end{verbatim}

It is important that you update regularly, as giFT is still in
development, and is being updated frequently.

\subsection{Configuring the Build Environment}

First things first, make sure you have recent versions of autoconf
(2.5x), automake (1.4) and libtool (1.4.x). You \textbf{will} get
fatal errors if you're using other versions. Remember to run
\verb|autogen.sh| again after upgrading one or more of those
programs.

\subsubsection{Installing the Zlib Compression Library}

Next, you need to enable zlib compression, so your shares list
will be compressed before sending it to your PARENT nodes. This
will reduce the amount of data sent over the OpenFT network
significantly. To enable compression you must have zlib 1.1.4
installed. If you are using Debian, type:

\begin{verbatim}
  $ su -c 'apt-get -y install zlib1g zlib1g-dev'
\end{verbatim}

If you're not using Debian, and have wget installed on your system, 
use the provided zlib.mak file to install zlib:

\begin{verbatim}
  $ cd giFT
  $ make -f zlib.mak
  $ su -c 'make -f zlib.mak install'
\end{verbatim}

zlib.mak installs zlib in the /usr/local directory.

Otherwise, you will need to install zlib 1.1.4 manually.
See zlib.mak for the required steps.

\subsubsection{Running autogen.sh}

Next, run '\verb|./autogen.sh|'.
'\verb|./autogen.sh|' looks for zlib in the /usr directory by default
as this the location package maintainers such as apt-get and rpm use.

If zlib is installed in the /usr directory, type:

\begin{verbatim}
  $ ./autogen.sh
\end{verbatim}

If zlib is installed in another directory, such as /usr/local (as
zlib.mak does), type:

\begin{verbatim}
  $ ./autogen.sh --with-zlib=/usr/local
\end{verbatim}

This will generate and execute '\verb|./configure|'.

\subsubsection{Perl errors}
giFT has support for Perl. This enables developers to 'hook'
scripts to giFT's code, because not all problems are meant to be
solved in C.  For this 'libperl' is used. If you get an error
saying "\texttt{/usr/bin/ld: cannot find -lperl}", you don't have
it.

To install libperl for Debian, do:

\begin{verbatim}
  $ su -c 'apt-get -y install libperl-dev'
\end{verbatim}

For now, you can also disable Perl. Just run '\verb|./autogen.sh|',
with the magic option:

\begin{verbatim}
  $ ./autogen.sh --disable-perl
\end{verbatim}

\subsubsection{Macro errors}
If you receive any errors (you can ignore all warnings), such as
"\texttt{'AM-PATH-GTK' not found in library}", "\texttt{'syntax
error near unexpected token `AM\_INIT\_AUTOMAKE(giFT,'}", or
"\texttt{Syntax error: word unexpected (expecting ")")}" you
either need to install the GTK development files (libgtk-dev on
debian, gtk-devel on rpmfind.net) or move the needed GTK macros
manually with:

\begin{verbatim}
  $ mv m4/support/gtk.m4 m4/
\end{verbatim}

and run '\texttt{./autogen.sh}' again.

\subsubsection{Enabling the GTK client}
The GTK client (giFT-fe) is disabled by default. If you really
want to try it (you'll have to fix it too, because it's currently
broken :-) run autogen.sh with the --enable-gtk-client option:

\begin{verbatim}
  $ ./autogen.sh --enable-gtk-client
\end{verbatim}

\subsection{Compiling}

To compile giFT and OpenFT, type:

\begin{verbatim}
  $ make
\end{verbatim}

\subsection{Installing}

To install the giFT daemon, giFT-setup, the OpenFT library and
giFT-shell, type:

\begin{verbatim}
  $ su -c 'make install'
\end{verbatim}

\subsection{Configuring giFT}

Before you can use giFT, you need to configure it by typing:

\begin{verbatim}
  $ giFT-setup
\end{verbatim}

giFT-setup will ask you several questions, and create the
configuration files based on your answers.  These files will be
created in the directory \texttt{\~{}/.giFT}. Make sure you
actually read all the information before answering.

\section{Running giFT}
Running giFT is easy:

\begin{verbatim}
  $ giFT
\end{verbatim}

That'll start the daemon and load all protocols (currently only
OpenFT). Because giFT is a \emph{daemon}, it will "just sit there",
like it doesn't do anything. If you really want to know, use
\verb|tail -f ~/.giFT/gift.log| to see what it is doing.

When you run giFT for the first time, it will most likely hog your
CPU and drive your harddrive crazy. This is because giFT will need
to calculate the md5sum (this process is called 'hashing') of all
your shared files to create a database. The next time you run giFT
only files that were added or changed will be hashed.

You can check out some info about the nodes you're connected to
and files you share with your browser at localhost:1216, or
another port you configured to be your nodepage. To find out what
your http port is, do 
\verb|grep http_port ~/.giFT/OpenFT/OpenFT.conf|. 

\subsection{GUI's}
To search for and download files you need a graphical user
interface (GUI).  At the moment, there are a few available. Please
see \url{http://gift.sourceforge.net/dev/clients.php} for a more
complete list. If you want to write your own, please check the
\href{http://gift.sourceforge.net/docs/?document=interface.html}{Interface
Protocol Documentation}. \textbf{NOTE: the Interface Protocol will
be redesigned very soon.}

\subsubsection{giFTcurs}
giFTcurs is an ncurses interface to giFT, written by weinholt and
chnix.  It is currently the suggested client, since it supports
most functions and just looks damn nice. For more information,
please see \url{http://giftcurs.sourceforge.net/}. 

To build giFTcurs, type:

\begin{verbatim}
  $ cvs -d:pserver:anonymous@cvs.giftcurs.sf.net:/cvsroot/giftcurs \
    co giFTcurs
  $ cd giFTcurs
  $ ./autogen.sh
  $ make
\end{verbatim}

\subsubsection{giFT-fe}
giFT-fe, is the official GUI.  It's GTK based but not maintained
at the moment.  If you would like to do that,
you're more than welcome.

\subsection{nodes file}
Ah yes, the ever popular 'nodes' file... Since many questions in
our IRC channel were asked because of this file, it gets a special
section here.
 
If you have problems connecting to OpenFT, remove
\texttt{\~{}/.giFT/OpenFT/nodes}. The default nodes file will be
copied to your home directory. 

\section{Miscellaneous}
A section for all things I don't know where to fit :-)

\subsection{Stats}
Since OpenFT works with HTTP, it also generates an Apache style
'\texttt{access.log}'.  With some tweaking you can make beautiful
overviews and graphs with your favorite log analyser / statistics
generator.  You can find the file in the directory
\texttt{\~{}/.giFT/access.log}.

\end{document}

% vim:tw=66:
