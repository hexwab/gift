% $Id: install.tex,v 1.9 2002/04/30 07:52:41 eelcol Exp $
% Copyright (c) 2001-2002, The giFT Project
% written by Eelco Lempsink (eelco@wideview.33lc0.net)

\documentclass[10pt]{article}

\usepackage{url}
\usepackage{hyperref}

\setlength{\parindent}{0pt}

\begin{document}

\begin{center}
\textsf{\textbf{\Huge{giFT}\\ \huge{Installation Guide}\\
\normalsize{Last update: \today}}}
\end{center}

\tableofcontents

\setlength{\parskip}{1.4ex}

\section{Introduction}
A small notice to begin with. Lots (and I mean lots) of people are coming to
our channel and ask about things covered in this document. Please note that
every line, every word and every letter in this document actually matters. If
you haven't read these instructions completely, and start asking dumb things,
you will be shot.

\section{Getting it}

Because giFT/OpenFT has not been released, you must retrieve the sources via 
CVS in order to use the program.

Go to the directory you want to download the sources and type (the \$ is used 
to indicate this is a shell command):

\begin{verbatim}
  $ cvs -d:pserver:anonymous@cvs.gift.sf.net:/cvsroot/gift login
\end{verbatim}

Just press enter when prompted for a password. You'll only have to do this
once, after that cvs will read your password information from \~{}/.cvspass.
Now type:

\begin{verbatim}
  $ cvs -z3 -d:pserver:anonymous@cvs.gift.sf.net:/cvsroot/gift co giFT
\end{verbatim}

This will download everything you need...

\subsection{Updating}
To update giFT to include the latest updates from CVS, type:

\begin{verbatim}
  $ cd giFT
  $ cvs update -Pd
\end{verbatim}

It is important that you update regularly, as giFT is still in development, 
and is being updated frequently.

\subsection{Configuring the Build Environment}

To configure the build scripts, type:

\begin{verbatim}
  $ cd giFT
  $ ./autogen.sh
\end{verbatim}

This will generate and execute '\verb|./configure|'.

If you receive the any errors (you can ignore all warnings), like 
"\texttt{'AM-PATH-GTK' not found in library}", "\texttt{'syntax error near
unexpected token `AM\_INIT\_AUTOMAKE(giFT,'}", or "\texttt{Syntax error: word
unexpected (expecting ")")}" you either need to install the GTK development
files (libgtk-dev on debian, gtk-devel on rpmfind.net) or move the needed GTK
macros manually with:

\begin{verbatim}
  $ mv m4/support/gtk.m4 m4/
\end{verbatim}

and run '\texttt{./autogen.sh}' again.

\subsection{Enabling the GTK client}
Currently, the GTK client (giFT-fe) is disabled by default. If you really want
to try it (it works, but not nearly as well as giFTcurs) run autogen.sh with
the --enable-gtk-client option:

\begin{verbatim}
  $ ./autogen.sh --enable-gtk-client
\end{verbatim}

\subsection{Compiling}

To compile giFT and OpenFT, type:

\begin{verbatim}
  $ make
\end{verbatim}

\subsection{Installing}

To install giFT, giFT-setup, the OpenFT library, the GTK GUI giFT-fe and the
giFT-shell, type:

\begin{verbatim}
  $ su -c 'make install'
\end{verbatim}

\subsection{Configuring giFT}

Before you can use giFT, you need to configure it by typing:

\begin{verbatim}
  $ giFT-setup
\end{verbatim}

\texttt{giFT-setup} will ask you several questions, and create the configuration 
files based on your answers.  
These files will be created in the directory \texttt{\~{}/.giFT}.


\section{Running giFT}
Running giFT is easy:

\begin{verbatim}
  $ giFT
\end{verbatim}

That'll start the daemon and load all protocols (currently only OpenFT).
The daemon will report all kind of status things, but you'll probably only
need them when giFT crashes. :-)

You can also check out some info about the nodes you're connected to and files
you share with your browser at localhost:1216, or another port you configured
to be your nodepage. To find out what your http port is, look for the line
containing \verb|http_start: http_port = [port]| just after starting giFT.

\subsection{GUI's}
To search for and download files you need a graphical user interface (GUI).  At
the moment, there are a few available. Please see
\url{http://gift.sourceforge.net/dev/clients.php} for a more complete list. If
you want to write your own, please check the
\href{http://gift.sourceforge.net/docs/?document=interface.html}{Interface
Protocol Documentation}.

\subsubsection{giFTcurs}
giFTcurs is an ncurses interface to giFT, written by weinholt and chnix. 
It is currently the suggested client, since it supports most functions and 
just looks damn nice. For more information, please see
\url{http://giftcurs.sourceforge.net/}. 

To build giFTcurs, type:

\begin{verbatim}
  $ cvs -d:pserver:anonymous@cvs.giftcurs.sf.net:/cvsroot/giftcurs \
    co giFTcurs
  $ cd giFTcurs
  $ ./autogen.sh
  $ make
\end{verbatim}

\subsubsection{giFT-fe}
giFT-fe, is the official GUI. 
It's GTK based and you already have it installed if you installed giFT.
Run giFT-setup to configure it.

\subsubsection{kift}
kift is a KDE-GUI.
For more information, please see \url{http://kift.sourceforge.net/}.

\subsubsection{giFT-shell}
giFT-shell, is malverians perl client, previously called gift-perl or 'TooT'
(for some mysterious reason :-). Its interface is gnut like, and it's an
excellent client for all command line junkies.  This client is installed by
default too.
 
\subsection{nodes file}
Ah yes, the ever popular 'nodes' file... Since a lot of questions in our IRC 
channel were asked because of this file, it gets a special section here.

Information about OpenFT nodes will be saved in the file
\texttt{\~{}/.giFT/OpenFT/nodes}. This information is vital if you want to
connect to the network, so a default nodes file is provided. That file is
installed in \texttt{/usr/local/share/giFT/OpenFT/nodes}.

If you have problems connecting to OpenFT, please try to remove your nodes file
(stop giFT first!) and restart giFT. The default file will be copied to your
home directory. If your nodes file is empty, giFT will do nothing.


\section{Miscellaneous}
A section for all things I don't know where to fit :-)

\subsection{Stats}
Since OpenFT works with HTTP, it also generates an Apache style 
'\texttt{access.log}'. 
With some tweaking you can make beautiful overviews and graphs with your 
favorite log analyser / statistics generator. 
You can find the file in the directory
\texttt{\~{}/.giFT/access.log}.

\end{document}
