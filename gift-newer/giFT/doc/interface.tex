% $Id: interface.tex,v 1.4 2002/04/26 18:32:07 eelcol Exp $
% Copyright (c) 2001-2002, The giFT Project
% written by Eelco Lempsink (eelco@wideview.33lc0.net)

\documentclass[10pt]{article}

\usepackage{url}
\setlength{\parindent}{0pt}

\begin{document}

\begin{center}
\textsf{\textbf{\Huge{giFT}\\ \huge{Interface Protocol Documentation}\\
\normalsize{Last update: \today}}}
\end{center}

\tableofcontents

\addtolength{\parskip}{1.4ex}

\section{Theory}
The giFT interface protocol was adopted so that the back-end was never a
required understanding for the user interface developer.  This unique feature
(jabber who?) allows any giFT UI developer to create a UI that can be used
unaltered if the user decides to load extra or even a completely different
protocol plugin.  The giFT project hopes that, in time, this interface protocol
will become a pseudo standard and will allow truly robust interfaces to be
built without any requirement on the network protocol it was intended for.

\section{Attached Connections}
The giFT interface protocol implements a scheme that allows manipulation on a
single socket via event identifiers or to use a unique socket connection to the
daemon for each event request. The recommendation is that a full-fledged user
interface will want to use the single socket interface and will thus require
the attach command.

Once a connection is properly attached (described below), the following
transaction will occur between the user interface and the daemon when
registering any event type:

Send the event:\\
\verb|  <EVENT_NAME [arg1="value"] [arg2="value" [arg3="value"]]>|

Receive the event identifer:\\ 
\verb|  <event action="EVENT_NAME" id="n"/>|

Event termination is received as (when the event is completed):\\
\verb|  <EVENT_NAME id="n"/>|\\
\emph{NOTE}: the GUI will eventually be able to "cancel" any event by sending
the above command to the daemon while the event is active. This functionality
is not currently implemented, however. 

\section{Protocol}
All commands are sent in the form \verb|<command [option1="value" [...]]/>|.
Each unique command must strictly send a "\verb|\n|" to indicate processing. It
is recommended that you use "\verb|\r\n|" (linefeed, linebreak).

Options between square brackets ([ and ]) are optional.

\subsection{attach}
Establishes an attached connection (see above)

\begin{tabular}{p{2.6cm}p{8.5cm}}
\verb|<attach|          & \\
\verb|    [profile]|    & Specifies which user profile to load preferences from. Currently unused. \\
\verb|    [client]|     & Authenticate your client by name. \\
\verb|    [version]|    & Register your clients version. \\
\verb|/>|               &
\end{tabular}

Upon attaching a new connection, you will receive all currently active
transfers. \emph{NOTE}: If no transfer is available, nothing will be sent back,
you must simply trust that it was done

\subsection{search}
Search queries and returned information.

\subsubsection{query}
Search for files:

\begin{tabular}{p{2.6cm}p{8.5cm}}
\verb|<search|          &  \\
\verb|    query|            & Search string \\
\verb|    [exclude]|        & Exclusion list \\
\verb|    [protocol]|       & Specify the protocol to use for searching. If none are present, all protocols are searched. \\
\verb|    [type]|    & Modifies search behavior for query (valid options are "user" or "hash". The query string must respectivly be an ip or a md5 hash) \\
\verb|    [realm]|          & Search realm, audio, video, mime types \\
\verb|    [csize]|          & Size constraint \\
\verb|    [ckbps]|          & Kilobits per second constraint \\
\verb|/>|               &
\end{tabular}

\subsubsection{result}
Results will be returned (between \verb|<event action="search" id="n"/>| and
\verb|<search id="n"/>|, like said above) as \verb|<item arg="val"/>|. Those
args are:

\begin{tabular}{p{3.1cm}p{8.0cm}}
\verb|<item|        &  \\
\verb|    [id]|     & Will only appear if you work with an attached connection. \\
\verb|    user|     & IP of node which has the file, this will become a name. \\
\verb|    node|     & IP of the search node which returned this result. \\
\verb|    href|     & Protocol-specific source string. You need this for \verb|<transfer addsource="string">| (see below). \\
\verb|    size|     & File size \\
\verb|    hash|     & Hash of the file \\
\verb|    availability| & Number of free upload slots the remote host has, can be 0, always 1 when the host has no upload limit. See the Notes for more information.\\
\verb|/>|           &
\end{tabular}

\subsection{transfer}
Downloading and uploading control and progress information.

\subsubsection{download}
Request a file download:

\begin{tabular}{p{3.9cm}p{7.2cm}}
\verb|<transfer|        &  \\
\verb|    action="download"|    & Tell giFT you want to download \\
\verb|    size|                 & Size of the file in bytes \\
\verb|    hash|                 & MD5 hash of the file \\
\verb|    save|                 & Filename to save as, do not include directory path \\
\verb|/>|           &
\end{tabular}

An event id will be returned (\verb|<event action="transfer" id="n"/>|). You'll
need this for the next step, adding sources for the download. This step is
required and must be handled by the GUI. giFT supports dynamic multiple source
downloads. Sources must be added or removed like this:

\begin{tabular}{p{2.6cm}p{8.5cm}}
\verb|<transfer|        &  \\
\verb|    id|           & Event identifer returned from the above transfer request \\
\verb|    user|         & Username (display name, currently IP) returned from the search request \\
\verb|    addsource|    & Protocol-specific source string returned from a search (\verb|href|) \\
\verb|      |\textbf{OR}  & \\
\verb|    delsource|    & Protocol-specific source string returned from a search (\verb|href|) \\
\verb|/>|               &
\end{tabular}

\subsubsection{upload}
Upon receiving an upload request, the daemon will send a transfer command to
all attached sockets so that the UI may monitor the progress.

\begin{tabular}{p{3.9cm}p{7.2cm}}
\verb|<transfer|    &  \\
\verb|    id|       & Event identifer pre-determined by the daemon \\
\verb|    action = "upload"| & To tell the GUI this is an upload \\
\verb|    user|     & Display string from the protocol, use this to show who is being uploaded to \\
\verb|    size|     & Total length of the requested segment \\
\verb|    href|     & Protocol-specific source string. Lists the remote host with the path of the requested file. \\
\verb|/>|               &
\end{tabular}

\subsubsection{progress}
During transfers the deamon will output information about the progress in a
\verb|<transfer/>| tag. The interval is 1 second. Id's are used to reconize
different tranfers.

\begin{tabular}{p{2.6cm}p{8.5cm}}
\verb|<transfer|    &  \\
\verb|    id|           & Event identifer \\
\verb|    transmit|     & Currently transmitted \\
\verb|    total|        & Total filesize \\
\verb|/>|           &
\end{tabular}

After the \verb|<transfer/>| tag, one or more \verb|<chunk/>| tags will follow.
They provide information about the progress of the different sources.

\begin{tabular}{p{2.6cm}p{8.5cm}}
\verb|<chunk|    &  \\
\verb|    id|           & Event identifer \\
\verb|    href|         & Protocol-specific source string \\
\verb|    start|        & Offset (in bytes) this chunk starts \\
\verb|    transmit|     & Currently transmitted \\
\verb|    total|        & Total size of the chunk \\
\verb|    status|       & Status of the chunk (\verb|Waiting|, \verb|Queued|, \verb|Active| or \verb|Dead|. Read the Notes for more information.) \\
\verb|/>|           &
\end{tabular}

When a transfer is finished an event termination tag will be sent \\
(\verb|<transfer id="n"/>|).
  
\subsubsection{cancel}
Cancel a transfer:

\begin{tabular}{p{3.9cm}p{7.2cm}}
\verb|<transfer|    &  \\
\verb|    action = "cancel"| & The transfer will be stopped and all related files removed \\
\verb|    id|       & Event identifer \\
\verb|/>|           &
\end{tabular}

Of course an event termination tag will be sent.

\subsection{share}
Update, hide, or show your shares.

\subsubsection{update}
Update your shares with:

\begin{tabular}{p{3.9cm}p{7.2cm}}
\verb|<share|   & \\
\verb|    [action="sync"]|    & Default action, not required \\
\verb|/>|       &
\end{tabular}

No event identifiers or other notifications will be returned.

\subsubsection{hide and show}
Hide or show your shares:

\begin{tabular}{p{3.9cm}p{7.2cm}}
\verb|<share|   & \\
\verb|    action="hide"|    & Hide your shares, search nodes will no longer return your files in their results. \\
\verb|      |\textbf{OR}    & \\
\verb|    action="show"|    & (Re-)show your shares again. \\
\verb|/>|       &
\end{tabular}

Again, no event identifiers or other notifications will be returned.
  
%\subsection{opt}
%Read and set options.
%
%\subsubsection{query}
%\begin{tabular}{p{2.6cm}p{8.5cm}}
%\verb|<opt|   & \\
%\verb|     [conf|   & Config section \\
%\verb|     [key|    & Config key (in section) \\
%\verb|      value]]| & Set key to value \\
%\verb|/>|     &
%\end{tabular}
%
%\subsubsection{result}
%\begin{tabular}{p{2.6cm}p{8.5cm}}
%\verb|<opt|   & \\
%\verb|     [conf|   & Config section \\
%\verb|     [key|    & Config key (in section) \\
%\verb|      value]]| & Value of key \\
%\verb|/>|     &
%\end{tabular}

\subsection{stats}
Statistical information.

\subsubsection{query}
Show stats:

\begin{tabular}{p{2.6cm}p{8.5cm}}
\verb|<stats|   & \\
\verb|     |    & No options \\
\verb|/>|       &
\end{tabular}

\subsubsection{result}
Returned information:

\begin{tabular}{p{2.6cm}p{8.5cm}}
\verb|<stats|   & \\
\verb|    id|   & Event identifier, only useful when attached \\
\verb|    protocol| & Show specific protocol. "local" means your local shares \\
\verb|    [status]| & (not if \verb|protcol="local"|) "Online" or "Offline" \\
\verb|    [users]|  & (not if \verb|protcol="local"|) Number of users on the network \\
\verb|    files|    & Number of shared files \\
\verb|    size|     & Total size of all shared content, expressed in GB (no floating point precision) \\
\verb|/>|       &
\end{tabular}

An event identifier and terminator will be sent when attached.

\subsection{quit}
Shutdown giFT.

\subsubsection{query}
\begin{tabular}{p{2.6cm}p{8.5cm}}
\verb|<quit|   & \\
\verb|     |    & No options \\
\verb|/>|       &
\end{tabular}

\subsubsection{result}
giFT will be cleanly shutdown.

\section{Notes}

\subsection{Availability}
The preferred method of displaying availability is to warn users somehow (or
optionally filter out) when a source has 0 availability.  I know everyone is
tempted to add up each availability tag but that's simply not a great idea. The
suggested method is adding "1" for each positive availability hit and ignoring
0. So if a search returns three sources with availability 3, 1 and 0, the
prefered reporting is '2' availability.

Keep in mind that if you tell giFT to download a source with availability=0
it'll keep trying until the source becomes available, so it's not really wise
to always hide the results from the user.

\subsection{Chunk stati}
A little more information about the different stati chunks can have.
\begin{itemize}
\item\verb|Waiting|\\
Downloading has not yet begun, no data has been received.
\item\verb|Queued|\\
Chunk is queued, local or remote. giFT will keep retrying.
\item\verb|Active|\\
Data has been received.
\item\verb|Dead|\\
Chunk has been removed (\verb|<transfer delsource/>|).
\end{itemize}

\section{Examples}
If you're still a little puzzled about how all this works, here are some
examples for some commands. Ignore the linebreaks, all commands must be sent as
one line.

\subsection{attach}
\begin{verbatim}
<attach/>
<attach profile="eelco" client="giFTcurs" version="0.3.2"/>
\end{verbatim}

\subsection{search}
Some queries:
\begin{verbatim}
<search query="blink 182"/>
<search query="blink 182" exclude="enema" realm="audio"/>
<search query="nekkid girls" csize="1048576-5242880"
    realm="image/jpeg">
<search query="234.234.234.234" type="user" ckbps="128"/>
\end{verbatim}

A Result:
\begin{verbatim}
<event action="search" id="182"/>
<item id="182" user="234.234.234.234" node="143.143.143.143"
    href="OpenFT://234.234.234.234:1357/mp3/good_music.mp3"
    size="23754241" hash="9edaaf435cf912ee1c31c9cb135942d6"
    availability="4"/>
<item id="182" user="214.214.214.214" node="143.143.143.143"
    href="OpenFT://214.214.214.214:2468/good_stuff.mpeg"
    size="329837321" hash="bce240de771d4d96e6384d7cd15f314e"
    availability="0"/>
<search id="182"/> 
\end{verbatim}

\subsection{transfer}

Download a file:
\begin{verbatim}
<transfer action="download" size="2972359"
    hash="ba7fb113e9947949d91b5a240983cdc6"
    save="i_want_this_name.mp3"/>
\end{verbatim}

An event id will be returned:
\begin{verbatim}
<event action="transfer" id="43"/>
\end{verbatim}

We use that to add sources:
\begin{verbatim}
<transfer id="43" user="32.132.32.132"
    addsource="OpenFT://32.132.32.132:5432/music/obscure.mp3"/>
<transfer id="43" user="22.12.22.12"
    addsource="OpenFT://100.10.100.10:1846/?listen_port=2222&
    request_host=22.12.22.12&
    request_file=/shares/normal_name.mp3"/>
\end{verbatim}

The download will start and progress will be reported:
\begin{verbatim}
<transfer id="43" transmit="0" total="2972359"/>
<chunk id="43"
    href="OpenFT://32.132.32.132:5432/music/obscure.mp3"
    start="0" transmit="0" total="14861179" status="Waiting"/>
<chunk id="43"
    href="OpenFT://100.10.100.10:1846/?listen_port=2222&
    request_host=22.12.22.12&
    request_file=/shares/normal_name.mp3" start="14861179"
    transmit="0" total="14861180" status="Waiting"/>
\end{verbatim}

The first chunk becomes active, but the second is queued:
\begin{verbatim}
<transfer id="43" transmit="4096" total="2972359"/>
<chunk id="43"
    href="OpenFT://32.132.32.132:5432/music/obscure.mp3"
    start="0" transmit="4096" total="14861179" status="Active"/>
<chunk id="43"
    href="OpenFT://100.10.100.10:1846/?listen_port=2222&
    request_host=22.12.22.12&
    request_file=/shares/normal_name.mp3" start="14861179"
    transmit="0" total="14861180" status="Queued"/>
\end{verbatim}

After a while we decide to cancel the queued chunk, so we send:
\begin{verbatim}
<transfer id="43" user="22.12.22.12"
    delsource="OpenFT://100.10.100.10:1846/?listen_port=2222&
    request_host=22.12.22.12&
    request_file=/shares/normal_name.mp3"/>
\end{verbatim}

The reporting will continue like this:
\begin{verbatim}
<transfer id="43" transmit="1048576" total="2972359"/>
<chunk id="43"
    href="OpenFT://32.132.32.132:5432/music/obscure.mp3"
    start="0" transmit="1048576" total="14861179"
    status="Active"/>
<chunk id="43"
    href="" start="14861179" transmit="0" total="14861180"
    status="Dead"/>
\end{verbatim}

Eventually... the download will finish
\begin{verbatim}
<transfer id="43"/> 
\end{verbatim}

\subsection{stats}
Request stats:
\begin{verbatim}
<stats/>
\end{verbatim}

Output:
\begin{verbatim}
<event action="stats" id="9"/>
<stats id="9" protocol="local" files="2486" size="9"/>
<stats id="9" protocol="OpenFT" status="Online" users="182"
    files="226619" size="2093"/>
<stats id="9"/>
\end{verbatim}

\end{document}
