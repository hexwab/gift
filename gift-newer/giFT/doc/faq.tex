% $Id: faq.tex,v 1.2 2002/04/22 21:08:38 eelcol Exp $
% Copyright (c) 2001-2002, The giFT Project
% written by Eelco Lempsink (eelco@wideview.33lc0.net)

\documentclass[10pt]{article}

\usepackage{url}
\usepackage{hyperref}

\setlength{\parindent}{0pt}

\newcommand{\question}[1]{\item\textbf{\emph{#1}}}

\begin{document}

\begin{center}
\textsf{\textbf{\Huge{giFT} \\ \huge{F.A.Q.} \\
\normalsize{Last update: \today}}}
\end{center}

\begin{center}
All the questions you shouldn't have asked in the IRC channel
\end{center}

\tableofcontents

\setlength{\parskip}{1.4ex}

\section{Generic}
\begin{itemize}

\question {What is giFT?}\\
giFT is a plugin system that can be used to connect to many different networks.
But for now OpenFT is the only network you can use. If you want to write a
plugin for other networks (Gnutella, OpenNAP), please contact us. 

\question {Where is your IRC channel?}\\
irc.openprojects.net, join channel \#giFT. 

\question {And, is it usable already?}\\
Yes it is, very usable, but also very unstable. If you want to help us finding
bugs, and you promise to update your copy daily, you can use it.

\question {Well then, how do I install and use giFT?}\\
That question is pretty much what the
\href{http://gift.sourceforge.net/docs/?document=install.html}{Installation
Guide} is all about... go read it.

\question {Why are you against the distributing of binaries?}\\
giFT is not ready. And if you're unable to build it yourself, it's not for you.
We don't want lots different and incompatible version floating around, because
they pollute our current test network.

\question {So, when do you release?}\\
When it's done. Once all feature users will want are implemented and all bugs
are squashed, and all documentation is up to date... then we'll probably
release. This could take a few weeks, it could take it few months. Just keep
checking :-).

\question {Why do you use a daemon?}\\
It creates an easy way for people to focus their time on frontends instead of
all of so much networking code. Also, if the daemon changes, all of the
frontends will still function the same as they had before. This also makes it
so we don't end up with 20 incompatible protocols flying around as standalone
clients.

\question {How did this all began?}\\
Development to create a Linux client to KaZaA was our initial goal. After a lot
of reverse engineering and packet sniffing, a group of talented developers
which became known as "the givers" stumbled onto our project (at the time
"kazaatux"). Within a few weeks, we had a working daemon that could connect
to and search the KaZaA network. Shortly after, KaZaA released a new version of
their client which eventually led to the breaking of what we now call giFT.
After the permanent departure of the givers, the giFT team started moving
development into a new direction; foremost was the desire to have a completely
open, completely free peer to peer network modeled in the image of FastTrack.
Thus OpenFT was born. giFT was moved to a new plugin architecture which would
allow the creation of plugins that would make it easy for one client to be
compatible with any number of networks. The first of these plugins were to be
OpenFT. 

\end{itemize}

\section{Contribute}
\begin{itemize}

\question {How can I help speed up development?}\\
Use your special skills! If you can code, design a nifty GUI or are specialized
in Public Relations, or whatever, drop into our IRC channel and ask what
\emph{you} can do.

\question {I've got an idea! What if...}\\
You're not the first one. If you want to discuss your idea or the way giFT
currently works, please
\href{http://lists.sourceforge.net/lists/listinfo/gift-openft}{subscribe to
gift-openft@lists.sourceforge.net}.
 
\question {Will giFT be ported to any other hosts?}\\
The program was initially designed for Linux, but the recent version compiles
on *BSD, Solaris, OS X, and Windows (natively!) too. You're always welcome to
port giFT to your platform of choice. Please contact us if you're successful. 
 
\question {Front-end Developers}\\
See the
\href{http://gift.sourceforge.net/docs/?document=interface.html}{Interface
Protocol Documentation}. We especially need a Windows GUI. 

\end{itemize}
 
\section{OpenFT} 
\begin{itemize}

\question {What's this OpenFT I've been hearing about?}\\
It's an Open Source "clone" of FastTrack currenty being worked on by the giFT
project. We welcome anyone with decentralized/p2p experience to drop by and
poke around at our design. 
 
\question {Can I use OpenFT without giFT?}\\
Yes, if you write a framework for it, like giFT. So, the real answer is: no.

\question {Is there documentation available on how OpenFT works?}\\
No, but the source code is pretty readable :-). The main problem is to find
someone motivated enough to write this document, if you want to do it, contact
us through IRC or use
\href{http://lists.sourceforge.net/lists/listinfo/gift-openft}{the mailing
list}.

\end{itemize}
 
\section{FastTrack (KaZaa, Morpheus, Grokster)}
\begin{itemize}

\question {What is this "FastTrack"?}\\
FastTrack is the company that licenses the library that KaZaA/Morpheus/Grokster
operates off of. 
 
\question {What's your connection with FastTrack?}\\
giFT 0.9 used the FastTrack network. FastTrack changed the encryption, end of
story. For the last 6 months we've been fully focused on the "new" giFT and

\question {Are you working on getting back into FastTrack's network?}\\
No. Although quite a few people made attempts to reverse engineer the new
encryption, nobody has succeeded, and nobody is working on it anymore because
OpenFT kicks so much ass nobody even remembers FastTrack. 
 
\end{itemize}

\end{document}
