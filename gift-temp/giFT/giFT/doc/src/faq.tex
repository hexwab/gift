% $Id: faq.tex,v 1.4 2003/02/25 10:22:02 eelcol Exp $
% Copyright (c) 2001-2002, The giFT Project
% written by Eelco Lempsink (eelco@wideview.33lc0.net)

% % % % % % % % % % % % % % % % % % % % % % % % % % % % % % % % %
% NOTE: anybody is more than welcome to come up with a better   %
% way of deviding the questions into sections                   %
% % % % % % % % % % % % % % % % % % % % % % % % % % % % % % % % %

\documentclass[10pt]{article}

\usepackage{url}
\usepackage{hyperref}

\setlength{\parindent}{0pt}

\newcommand{\question}[1]{\item\textbf{\emph{#1}}}

\begin{document}

\begin{center}
\textsf{\textbf{\Huge{giFT} \\ \huge{F.A.Q.} \\
\normalsize{Last update: \today}}}
\end{center}

\begin{center}
All the questions you should have read before you asked them
\end{center}

\tableofcontents

\setlength{\parskip}{1.4ex}

\section{Generic}
\begin{itemize}

\question {What is giFT?}\\
giFT is a plugin system that can be used to connect to many
different networks.  But for now OpenFT is the only network you
can use. More information about the giFT project is in the
\href{http://gift.sourceforge.net/docs.php?document=whatis.html}{What
is the giFT project?} document. If you want to write a plugin for
other networks (Gnutella, OpenNAP), please contact us.

\question {And, is it usable already?}\\
Yes it is, very usable, but also very unstable. If you want to
help us finding bugs, and you promise to update your copy daily,
you may use it.

\question {Why must I keep my copy up to date?}\\
Mostly because of OpenFT (see below). The way the network works
changes a lot, and you can significantly pollute the network if
you fail to update. Please keep a close eye on the CVS, and update
daily, or at least once every three days.

\question {Well then, how do I install and use giFT?}\\
That question is pretty much what the
\href{http://gift.sourceforge.net/docs.php?document=install.html}{Installation
Guide} is all about... go read it.

\question {Does giFT work on [platform]?}\\
giFT/OpenFT works on several platform.  See the doc/ directory for
installation instructions for Windows and MacOS X.

\question {Why are you against the distributing of binaries?}\\
giFT isn't ready for that. And if you're unable to build it
yourself, it's not for you.  We don't want lots different and
incompatible version floating around, because they pollute our
current test network.

\question {So, when do you release?}\\
When it's done. Once all feature users will want are implemented
and all bugs are squashed, and all documentation is up to date...
then we'll probably release. This could take a few weeks, it could
take it few months. Just keep checking :-).

\question {Why do you use a daemon?}\\
It creates an easy way for people to focus their time on frontends
instead of all of so much networking code. Also, if the daemon
changes, all of the frontends will still function the same as they
had before. This also makes it so we don't end up with 20
incompatible protocols flying around as standalone clients.

\question {Why the funny/boring/strange name?}\\
That's kind of historical. When giFT was still related to
FastTrack (see below), it used to mean 'generic interface to
FastTrack' (explains the capitalization) or 'giFT isn't
FastTrack'. A while after we moved away from FastTrack, and
started developing OpenFT, "nobody" on the forums suggested to
change the name to 'GNU Internet File Transfer', and we liked it,
so we changed the name.  A while after that, the name was changed
again, because we aren't really a GNU project, to 'giFT: Internet
File Transfer', and so it will probably stay\ldots

\question {How did this all begin?}\\
Development to create a Linux client to KaZaA was our initial
goal. After a lot of reverse engineering and packet sniffing, a
group of talented developers which became known as "the givers"
stumbled onto our project (at the time "kazaatux"). \\
Within a few weeks, we had a working daemon that could connect to
and search the KaZaA network. Shortly after, KaZaA released a new
version of their client which eventually led to the breaking of
what we now call giFT. \\
After the permanent departure of the givers, the giFT team started
moving development into a new direction; foremost was the desire
to have a completely open, completely free peer to peer network
modeled in the image of FastTrack. \\
Thus OpenFT was born. giFT was moved to a new plugin architecture
which would allow the creation of plugins that would make it easy
for one client to be compatible with any number of networks. The
first of these plugins were to be OpenFT.

\end{itemize}

\section{Problems}
\begin{itemize}

\question {Hey, giFT won't connect! (Before I updated it was
fine)}\\
Aha, first of all, you mean \emph{OpenFT} won't connect.
Secondly, the difference between the current version you're
running and your previous version is a major protocol change.
Because OpenFT is still under heavy development newer versions of
the protocol are not backwards compatible.  Removing your nodes
file will usually help. (See the 'nodes file' section of the
\href{http://gift.sourceforge.net/docs.php?document=install.html}{Installation
Guide} for more information).

\question {Huh? Where did all those users go? (Before I updated
there were tons of 'em)}\\
Like the previous question, this probably has to do with an OpenFT
protocol change.  If all our users would update daily, this
wouldn't be a problem.  The only solution is to be patient (and
helping us with telling users why they should update daily ;-).

\question {Aargh, giFT crashed! What now?}\\
You've most likely found a bug. Make sure you're using the most
recent CVS version (it could have been fixed already), and know
how to use gdb (GNU Debugger). It's wise to
\href{http://wand.cs.waikato.ac.nz/~dhtrl1/pl5/gdb.html}{do}
\href{http://web.mit.edu/sipb-iap/unixsoftdev/www/gdb.html}{some}
\href{http://users.actcom.co.il/~choo/lupg/tutorials/debugging/debugging-with-gdb.html}{homework}
prior to reporting bugs. Otherwise, you probably won't be very
helpful. (Note: the 'homework' pages aren't directly applicable to
giFT, but meant to learn to use gdb.)

\question {Help! Why does giFT almost bring my system to a
halt?}\\
Ah, you're running giFT for the first time\ldots  giFT is
calculating the md5sum (this process is called 'hashing') of all
your shared files to create a database. The next time you run giFT
only files that were added or changed will be hashed.

\question {Hmm, I get strange errors when trying to install,
what's up?}\\
Most likely you haven't read the
\href{http://gift.sourceforge.net/docs.php?document=install.html}{Installation
Guide}. You'll have to read it anyway, but the thing that usually works is
making sure the needed software is up to date (and updating it if
it isn't). Make sure you have autoconf 2.5x, automake 1.4 and
libtool 1.4.x. And read the damn
\href{http://gift.sourceforge.net/docs.php?document=install.html}{Installation
Guide}

\end{itemize}

\section{Contribute}
\begin{itemize}

\question {How can I help speed up development?}\\
Use your special skills! If you can code, design a nifty GUI or
are specialized in Public Relations, or whatever, drop into our
IRC channel and ask what \emph{you} can do.  Also, clear
bugreports, patches and more documentation are welcome.

\question {I've got an idea! What if...}\\
You're not the first one. If you want to discuss your idea or the
way giFT/OpenFT works, please
\href{http://lists.sourceforge.net/lists/listinfo/gift-openft}{subscribe
to gift-openft@lists.sourceforge.net}.

\question {Will giFT be ported to any other hosts?}\\
The program was initially designed for Linux, but the recent
version compiles on *BSD, Solaris, OS X, and Windows (natively)
too. You're always welcome to port giFT to your platform of
choice. Please contact us if you're successful.

\question {Front-end Developers}\\
See the
\href{http://gift.sourceforge.net/docs.php?document=interface.html}{Interface
Protocol Documentation}. There are already quite a few clients
available, \href{http://gift.sourceforge.net/dev/clients.php}{our
client page} lists the ones that work. Searching for 'giFT' on
SourceForge will find you some more giFT related projects. Please
check first if you can help an other project, before starting your
own.

\question {I want to make an OpenFT client!}\\
No, you don't. You want to make a giFT client. Through giFT,
OpenFT can be used. Otherwise you'll have to include a framework
just like giFT in your client, and that would be like reinventing
the wheel. Please read
\href{http://gift.sourceforge.net/docs.php?document=whatis.html}{What
is the giFT project?} for more info.

\end{itemize}

\section{OpenFT}
\begin{itemize}

\question {What's this OpenFT I've been hearing about?}\\
It's an Open Source "clone" of FastTrack being worked on by the
giFT project. We welcome anyone with decentralized/p2p experience
to drop by and poke around at our design.

\question {Can I use OpenFT without giFT?}\\
Yes, if you write a framework for it, like giFT. So, the real
answer is: no. If you're still convinced you want to write a new
framework, maybe even in an other language than C, like Java, C\#
or whatever, please contact us, and we'll talk you out of it.

\question {Is there documentation available on how OpenFT works?}\\
No, but the source code is pretty readable :-). The main problem
is to find someone motivated enough to write this document, if you
want to do it, contact us through 
\href{http://lists.sourceforge.net/lists/listinfo/gift-openft}{the
mailing list}.

\question {Hey, I can download files with my browser from OpenFT
hosts, is OpenFT a webserver?}\\
Uh, no. OpenFT uses an incomplete implementation of the HTTP
protocol for file transfers from non-firewalled hosts.
Theoretically you can use it as a webserver for static files, but
if you do so, you've missed the point :-)

%\question {Does OpenFT mean OpenFastTrack?}\\
%No, figure it out :-p
% we need a new name anyway...

\end{itemize}

\section{FastTrack (KaZaa, Morpheus, Grokster)}
\begin{itemize}

\question {What is this "FastTrack"?}\\
FastTrack is the company that licenses the library that
KaZaA/Grokster (Morpheus uses Gnutella now) operates off of

\question {What's your connection with FastTrack?}\\
giFT 0.9.x used the FastTrack network. FastTrack changed the
encryption, end of story. Since October 2001 we've been fully
focused on the "new" giFT and OpenFT. \textbf{Don't} ask us for a
FastTrack client, just forget about it.

\question {Are you working on getting back into FastTrack's network?}\\
No. Although quite a few people made attempts to reverse engineer
the new encryption, nobody has succeeded, and nobody is working on
it anymore because OpenFT kicks so much ass nobody even remembers
FastTrack.

\end{itemize}

\end{document}

% vim:tw=66:
